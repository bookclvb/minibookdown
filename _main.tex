% Options for packages loaded elsewhere
\PassOptionsToPackage{unicode}{hyperref}
\PassOptionsToPackage{hyphens}{url}
%
\documentclass[
  openany]{book}
\usepackage{amsmath,amssymb}
\usepackage{iftex}
\ifPDFTeX
  \usepackage[T1]{fontenc}
  \usepackage[utf8]{inputenc}
  \usepackage{textcomp} % provide euro and other symbols
\else % if luatex or xetex
  \usepackage{unicode-math} % this also loads fontspec
  \defaultfontfeatures{Scale=MatchLowercase}
  \defaultfontfeatures[\rmfamily]{Ligatures=TeX,Scale=1}
\fi
\usepackage{lmodern}
\ifPDFTeX\else
  % xetex/luatex font selection
\fi
% Use upquote if available, for straight quotes in verbatim environments
\IfFileExists{upquote.sty}{\usepackage{upquote}}{}
\IfFileExists{microtype.sty}{% use microtype if available
  \usepackage[]{microtype}
  \UseMicrotypeSet[protrusion]{basicmath} % disable protrusion for tt fonts
}{}
\makeatletter
\@ifundefined{KOMAClassName}{% if non-KOMA class
  \IfFileExists{parskip.sty}{%
    \usepackage{parskip}
  }{% else
    \setlength{\parindent}{0pt}
    \setlength{\parskip}{6pt plus 2pt minus 1pt}}
}{% if KOMA class
  \KOMAoptions{parskip=half}}
\makeatother
\usepackage{xcolor}
\usepackage{longtable,booktabs,array}
\usepackage{calc} % for calculating minipage widths
% Correct order of tables after \paragraph or \subparagraph
\usepackage{etoolbox}
\makeatletter
\patchcmd\longtable{\par}{\if@noskipsec\mbox{}\fi\par}{}{}
\makeatother
% Allow footnotes in longtable head/foot
\IfFileExists{footnotehyper.sty}{\usepackage{footnotehyper}}{\usepackage{footnote}}
\makesavenoteenv{longtable}
\usepackage{graphicx}
\makeatletter
\def\maxwidth{\ifdim\Gin@nat@width>\linewidth\linewidth\else\Gin@nat@width\fi}
\def\maxheight{\ifdim\Gin@nat@height>\textheight\textheight\else\Gin@nat@height\fi}
\makeatother
% Scale images if necessary, so that they will not overflow the page
% margins by default, and it is still possible to overwrite the defaults
% using explicit options in \includegraphics[width, height, ...]{}
\setkeys{Gin}{width=\maxwidth,height=\maxheight,keepaspectratio}
% Set default figure placement to htbp
\makeatletter
\def\fps@figure{htbp}
\makeatother
\setlength{\emergencystretch}{3em} % prevent overfull lines
\providecommand{\tightlist}{%
  \setlength{\itemsep}{0pt}\setlength{\parskip}{0pt}}
\setcounter{secnumdepth}{5}
\ifLuaTeX
  \usepackage{selnolig}  % disable illegal ligatures
\fi
\usepackage{bookmark}
\IfFileExists{xurl.sty}{\usepackage{xurl}}{} % add URL line breaks if available
\urlstyle{same}
\hypersetup{
  pdftitle={Coptic Stitch Bookbinding},
  pdfauthor={Emily Coxe},
  hidelinks,
  pdfcreator={LaTeX via pandoc}}

\title{Coptic Stitch Bookbinding}
\author{Emily Coxe}
\date{}

\begin{document}
\maketitle

{
\setcounter{tocdepth}{1}
\tableofcontents
}
\chapter{Textbook content under construction}\label{textbook-content-under-construction}

Hello, world!!!

\chapter{Why Use Coptic Stitch?}\label{why-use-coptic-stitch}

When there are so many ways to create books, you might wonder why you'd choose this notoriously difficult method. Here are a few reasons:

\begin{itemize}
\tightlist
\item
  This is a glue-free method.
\item
  The binding looks beautiful and you can leave it exposed.
\item
  Coptic bindings tend to be looser, which allows your pages to lie totally flat when you open the book. This is great for sketchbooks!
\item
  Sometimes it's fun to do things the hard way.
\end{itemize}

There are many visual tutorials on coptic stitch available online.
This is a very simple instructional manual, but feel free to reference the wealth of other resources available for further guidance.
If you enjoy this method and would like \emph{more} techniques for bookbinding, please check out the Resources chapter.
In particular, if you enjoy sewing bindings, check out Keith Smith's series {[}Non-Adhesive Binding{]} (\url{https://keithsmithbooks.com/Vol-I-NAB}).

\chapter{Assemble and Punch Your Pages}\label{assemble-and-punch-your-pages}

For this step, you can use pre-cut pages or cut your own to a size of your choosing.

\section{Folios and Signatures}\label{folios-and-signatures}

A \emph{folio} is one sheet of paper folded precisely in half.\\
A \emph{signature} is a stack of several signatures nestled together.

For this book, we will use five signatures of three pages each, so you need 15 folios (or 15 sheets of paper).
1. Stack 3 pages together and carefully fold the whole thing in half, creasing the fold with a bone folder or other flat object. Congratulations, you have a signature!
2. Take a paper awl and punch 5 holes in the crease through all 3 sheets. You should probably measure before you do this so that they are equidistant and replicable ;)
3. Repeat with four more stacks of three sheets each.

When you stack all five signatures together, the holes should more or less align at the folded edges.

\end{document}
